% Provides macros manipulating strings of tokens.
\RequirePackage{xstring}

% Store the jobname as a string with category 11 characters.
\edef\normaljobname{\expandafter\scantokens\expandafter{\jobname\noexpand}}
\StrBetween{\normaljobname}{asgmt-}{-q}[\assignmentnumber]
\StrBehind{\normaljobname}{-q-}[\questionnumber]

\documentclass[
  coursecode={MTHE 455},
  assignmentname={Assignment \assignmentnumber},
  studentnumber=20053722,
  name={Bryan Hoang},
  draft,
  % final,
]{
  ltxanswer%
}

\usepackage{bch-style}

\begin{document}
  \begin{questions}
    \setcounter{question}{\questionnumber}
    \addtocounter{question}{-1}
    \question[4]{}
    \begin{solution}
      \begin{proof}
        \begin{proofpart}
          (Irreducibility)

          Let \(i,j\in S\) where \(S\) is the state space of \(X_{n}:n\ge0\). WLOG, suppose that \(i<j\) and let \(n=j-i>0\). Then
          \begin{align*}
            p_{ij}^{(n)} &= P_{i}(X_{n}=j)              \\
                         &= \prod_{k=i}^{j-1} p_{k,k+1} \\
                         &> 0.
          \end{align*}
          Thus, \(i \to j\). In the other direction, we have
          \begin{align*}
            p_{ji}^{(n)} &= P_{j}(X_{n}=i)              \\
                         &= \prod_{k=i}^{j-1} p_{k+1,k} \\
                         &> 0.
          \end{align*}
          Thus, \(j \to i\). Then for arbitrary \(i,j\), we have that \(i \leftrightarrow j\). Therefore, the Markov chain is irreducible.
        \end{proofpart}
        \begin{proofpart}
          (Transience)

          Suppose that the Markov chain is recurrent. Then
          \begin{equation*}
            P_{1}(X_{n}=0\ \text{for some}\ n) = 1.
          \end{equation*}
          But the result from Question 3 in Assignment 2 is that
          \begin{align*}
            P_{1}(X_{n}\ge1, \forall n\ge0)                &> 0 \\
            \Rightarrow P_{1}(X_{n}=0\ \text{for some}\ n) &< 1
          \end{align*}
          which leads to a contradiction. Therefore, the Markov chain must be transient.
        \end{proofpart}
      \end{proof}
    \end{solution}
  \end{questions}
\end{document}
