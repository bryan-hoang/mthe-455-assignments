% Provides macros manipulating strings of tokens.
\RequirePackage{xstring}

% Store the jobname as a string with category 11 characters.
\edef\normaljobname{\expandafter\scantokens\expandafter{\jobname\noexpand}}
\StrBetween{\normaljobname}{asgmt-}{-q}[\assignmentnumber]
\StrBehind{\normaljobname}{-q-}[\questionnumber]

\documentclass[
  coursecode={MTHE 455},
  assignmentname={Assignment \assignmentnumber},
  studentnumber=20053722,
  name={Bryan Hoang},
  draft,
  % final,
]{
  ltxanswer%
}

\usepackage{bch-style}

\begin{document}
  \begin{questions}
    \setcounter{question}{\questionnumber}
    \addtocounter{question}{-1}
    \question[7]
    Let \(S=\{+1,-1\}^{d}\) denote the state space of \(X_{n}\).
    \begin{parts}
      \part{}
      \begin{solution}
        \begin{proof}
          For \(s,s^{*}\in S\), let \(k\) be the number of components the two states (co-ordinates) \(s\) and \(s^{*}\) differ by.

          Then by flipping one component at a time, \(\exists \{s_{i}:i\in[0,k]\}\) such that
          \begin{align*}
            s_{0} &= s      \\
            s_{k} &= s^{*},
          \end{align*}
          and that for all \(i\in[k-1]\), \(s_{i}\) and \(s_{i+1}\) differ in only one component.
          Then
          \begin{align*}
            P_{s}(X_{k}=s^{*}) &\ge P_{s}(X_{1}=s_{1},\dotsc,X_{k}=s_{k}) \\
                               &\ge \frac{1}{d^{k}}                       \\
                               &> 0.
          \end{align*}
          Therefore, \(\{X_{n}:n\ge0\}\) is irreducible.
        \end{proof}
      \end{solution}

      \part{}
      \begin{solution}
        For the state \(\mathbf{i}\in S\), to return to \(\mathbf{i}\), each component of the co-ordinate needs to be flipped an even number of times. Then \(\forall s\in S, \forall n\in\Z_{\ge1}\),
        \begin{equation*}
          P_{\mathbf{i}}(X_{2n-1}=s) = 0.
        \end{equation*}
        But,
        \begin{align*}
          P_{\mathbf{i}}(X_{2}=s) &= \frac{1}{d} \\
                                  &> 0.
        \end{align*}
        Therefore, the period of state \(\mathbf{i}\) is \fbox{2}.
      \end{solution}

      \part{}
      \begin{solution}
        \begin{claim}
          Let \(\pi_{s}=\frac{1}{2^{d}}, \forall s\in S\) and \(\pi=(\pi_{s})_{s\in S}\). Then \(\pi\) is the unique invariant distribution.
        \end{claim}
        \begin{proof}
          \(\forall s \in S\), let \(s_{-i}\) be the state ahcienved after flipping the \(i\)th component of \(s\) for \(i\in[1,d]\). Then
          \begin{align*}
            p_{s_{-i},s} &= \frac{1}{d} & &\forall i\in[1,d]                                 \\
            \intertext{and}
            p_{s^{*},s}  &= 0           & &\text{if}\ s^{*} \notin \{s_{-1},\dotsc,s_{-d}\}.
          \end{align*}
          Then \(\forall s \in S\),
          \begin{align*}
            \sum_{s^{*} \in S} \pi_{s^{*}} p_{s^{*},s} &= \sum_{i=1}^{d} \frac{1}{2^{d}} \cdot \frac{1}{d} \\
                                                       &= \frac{1}{2^{d}}                                  \\
                                                       &= \pi_{s}
          \end{align*}
          which means that \(\pi\) is an invariant distribution. Since the chain is irreducible by part~(\ref{part@4@1}), the invariant distribution is unique.
        \end{proof}
      \end{solution}

      \part{}
      \begin{solution}
        Let \(T_{\mathbf{i}}^{(1)}=\inf\{n\in\Z_{\ge1}:X_{n}=\mathbf{i}\}\) be the first passage time to \(\mathbf{i}\). From a corollary discussed in class, we have that
        \begin{equation*}
          \pi_{\mathbf{i}} = \frac{1}{\E_{\mathbf{i}}{T_{\mathbf{i}}^{(1)}}}
        \end{equation*}
        But since \(\pi_{\mathbf{i}}=\frac{1}{2^{d}}\) from part~(\ref{part@4@3}), we have that
        \begin{equation*}
          \E_{\mathbf{i}}{T_{\mathbf{i}}^{(1)}} = 2^{d}.
        \end{equation*}
        Therefore, the expected number of steps until the particle returns it \(\mathbf{i}\) is \(\boxed{2^{d}}\).
      \end{solution}

      \part{}
      \begin{solution}
        \(\forall i \in S\), define
        \begin{equation*}
          \gamma_{i}^{\mathbf{i}} = \E_{\mathbf{i}}*{\sum_{n=0}^{T_{\mathbf{i}}^{(1)}} \mathbb{1}_{\{X_{n}=i\}}}.
        \end{equation*}
        Then by Theorems discussed in class, \(\gamma^{\mathbf{i}}\) is an invariant measure and is identical to \(\pi\) up to scale. Thus,
        \begin{align*}
          \gamma_{\mathbf{o}}^{\mathbf{i}} &= \gamma_{\mathbf{i}}^{\mathbf{i}} \\
                                           &= 1.
        \end{align*}
        Therefore, the expected number of visits to \(\mathbf{o}\) until the particle returns to \(\mathbf{i}\) is \(\boxed{1}\).
      \end{solution}
    \end{parts}
  \end{questions}
\end{document}
