% Provides macros manipulating strings of tokens.
\RequirePackage{xstring}

% Store the jobname as a string with category 11 characters.
\edef\normaljobname{\expandafter\scantokens\expandafter{\jobname\noexpand}}
\StrBetween{\normaljobname}{asgmt-}{-q}[\assignmentnumber]
\StrBehind{\normaljobname}{-q-}[\questionnumber]

\documentclass[
  coursecode={MTHE 455},
  assignmentname={Assignment \assignmentnumber},
  studentnumber=20053722,
  name={Bryan Hoang},
  draft,
  % final,
]{
  ltxanswer%
}

\usepackage{bch-style}

\begin{document}
  \begin{questions}
    \setcounter{question}{\questionnumber}
    \addtocounter{question}{-1}
    \question[7]\
    \begin{parts}
      \part{}
      \begin{solution}
        \begin{proof}
          \(\forall n\in\Z\),
          \begin{equation*}
            \tilde{P}^{n} = \sum_{k=0}^{n} \binom{n}{k} \alpha^{k} (1-\alpha)^{n-k} P^{k},
          \end{equation*}
          where \(P^{0}=I_{N}\). \(\because P\) has positive entries, \(\forall k\in[1,n]\),
          \begin{equation*}
            \tilde{P}^{n} \ge \binom{n}{k} \alpha^{k} (1-\alpha)^{n-k} P^{k},
          \end{equation*}
          where the inequality is element-wise.

          Then \(\forall i,j\in[1,N]\), the irreducibility of \(P\) implies that \(\exists n_{i,j}\in\Z_{\ge1}:P_{i,j}^{n_{i,j}}>0\).

          Let \(n^{'}=\max_{i,j\in[1,N]}n_{i,j}\). Then \(\forall i,j\in[1,N], \forall n\ge n^{'}, \forall k\in[1,n]\),
          \begin{align*}
            \tilde{P}_{i,j}^{n} &\ge \binom{n_{i,j}}{k} \alpha^{n_{i,j}} (1-\alpha)^{n-n_{i,j}} P^{n_{i,j}} \\
                                &> 0
          \end{align*}
          which implies that \(\tilde{P}\) is aperiodic and irreducible.
        \end{proof}
      \end{solution}

      \part{}
      \begin{solution}
        \begin{proof}
          Suppose that \(\pi\) is an invariant distribution for \(P\). Then
          \begin{align*}
            \pi\tilde{P} &= \pi(\alpha P + (1-\alpha) I_{N})        & &\text{by the definition of \(\tilde{P}\)} \\
                         &= \alpha(\pi P) + (1 - \alpha)(\pi I_{N})                                              \\
                         &= \pi                                     & &\because\ \pi P=\pi.
          \end{align*}
          Therefore, \(\pi\) is also an invariant distribution for \(\tilde{P}\).
        \end{proof}
      \end{solution}
    \end{parts}
  \end{questions}
\end{document}
