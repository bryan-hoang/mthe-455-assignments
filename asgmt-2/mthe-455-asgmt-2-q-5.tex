% Provides macros manipulating strings of tokens.
\RequirePackage{xstring}

% Store the jobname as a string with category 11 characters.
\edef\normaljobname{\expandafter\scantokens\expandafter{\jobname\noexpand}}
\StrBetween{\normaljobname}{asgmt-}{-q}[\assignmentnumber]
\StrBehind{\normaljobname}{-q-}[\questionnumber]

\documentclass[
  coursecode={MTHE 455},
  assignmentname={Assignment \assignmentnumber},
  studentnumber=20053722,
  name={Bryan Hoang},
  % final,
]{
  ltxanswer%
}

\usepackage{bch-style}

\begin{document}
  \begin{questions}
    \setcounter{question}{\questionnumber}
    \addtocounter{question}{-1}
    \question[5]\
    \begin{parts}
      \part{}
      \begin{solution}
        \begin{proof}
          Let \(Y=F(X)\) with the CDF \(F_{Y}\) and let \(F_{U}\) be the CDF of a \(U(0,1)\) random variable, where
          \begin{equation*}
            F_{U}(u) = \begin{cases}
              0 & \text{for } u < 0,        \\
              x & \text{for } u \in [0, 1], \\
              1 & \text{for } u > 1,
            \end{cases}
          \end{equation*}
          It is sufficient to show that \(Y \sim U(0,1)\) by showing that \(F_{Y} = F_{U}\).

          For \(y < 0\), \(F_{Y}(y) = P(F(X) \le y) = P(F(X) \le 0) = 0\) by the definition of a CDF.\@

          For \(y > 1\), \(F_{Y}(y) = P(F(X) \le y) = P(F(X) \le 1) = 1\) by the definition of a CDF.\@

          For \(y \in [0,1]\),
          \begin{align*}
            F_{Y}(y) &= P(F(X) \le y)      \\
                     &= P(X \le F^{-1}(y)) \\
                     &= F(F^{-1}(y))       \\
                     &= y
          \end{align*}
          since \(F\) is continuous and strictly increasing, and thus invertible. Thus, we have
          \begin{align*}
            F_{Y}(y) &= \begin{cases}
                          0 & \text{for } y < 0,        \\
                          y & \text{for } y \in [0, 1], \\
                          1 & \text{for } y > 1,
                        \end{cases} \\
                     &= F_{U}(y)
          \end{align*}
          Therefore, \(Y = F(X) \sim U(0,1)\).
        \end{proof}
      \end{solution}

      \part{}
      \begin{solution}
        \begin{proof}
          Let \(Y = F^{-1}(U)\) with CDF \(F_{Y}\) and let \(F_{U}\) be the CDF of \(U \sim U(0,1)\). Then it is sufficient to show that \(F_{Y} = F\).

          For \(y \in \mathbb{R}\), we have
          \begin{align*}
            F_{Y}(y) &= P(F^{-1}(U) \le y)                                                               \\
                     &= P(U \le F(y))                                                                    \\
                     &= F_{U}(F(y))                                                                      \\
                     &= F(y)               & &\because F(y) \in [0,1]\ \text{by the definition of a CDF}
          \end{align*}
          Thus, we have that \(Y = F^{-1}(U) \sim X\).
        \end{proof}
      \end{solution}
    \end{parts}
  \end{questions}
\end{document}
