% Provides macros manipulating strings of tokens.
\RequirePackage{xstring}

% Store the jobname as a string with category 11 characters.
\edef\normaljobname{\expandafter\scantokens\expandafter{\jobname\noexpand}}
\StrBetween{\normaljobname}{asgmt-}{-q}[\assignmentnumber]
\StrBehind{\normaljobname}{-q-}[\questionnumber]

\documentclass[
  coursecode={MTHE 455},
  assignmentname={Assignment \assignmentnumber},
  studentnumber=20053722,
  name={Bryan Hoang},
  % final,
]{
  ltxanswer%
}

\usepackage{bch-style}

\begin{document}
  \begin{questions}
    \setcounter{question}{\questionnumber}
    \addtocounter{question}{-1}
    \question[5]{}
    \begin{solution}
      Let \(X_{n}\) be the vertex the particle is on at step \(n\). Then \(\{X_{n} : n \ge 0\}\) is a Markov chain with the state space \(S=\{A,B,C,D,E\}\) and the transition matrix
      \begin{equation*}
        P = \begin{bmatrix}
          0           & \frac{1}{2} & \frac{1}{2} & 0           & 0           \\
          \frac{1}{2} & 0           & \frac{1}{2} & 0           & 0           \\
          \frac{1}{4} & \frac{1}{4} & 0           & \frac{1}{4} & \frac{1}{4} \\
          0           & 0           & \frac{1}{2} & 0           & \frac{1}{2} \\
          0           & 0           & \frac{1}{2} & \frac{1}{2} & 0
        \end{bmatrix}
      \end{equation*}
      Denote by \(F\) the event that the Markov chain hits the state \(E\) before it hits \(B\), and denote
      \begin{equation*}
        f_{i} = \mathbb{P}_{i}(F), \qquad \text{for } i = A, B, C, D, E.
      \end{equation*}
      Then conditional on \(X_{0}\), we have
      \begin{align*}
        f_{A} &= \frac{1}{2}f_{B} + \frac{1}{2}f_{C}                                       \\
        f_{B} &= 0                                                                         \\
        f_{C} &= \frac{1}{4}f_{A} + \frac{1}{4}f_{B} + \frac{1}{4}f_{D} + \frac{1}{4}f_{E} \\
        f_{D} &= \frac{1}{2}f_{C} + \frac{1}{2}f_{E}                                       \\
        f_{E} &= 1
      \end{align*}
      Thus, by solving the system of equations, we have \(f_{A}=\frac{1}{4}\), \(f_{C}=\frac{1}{2}\), and \(f_{D}=\frac{3}{4}\).

      In particular, if \(X_{0}=A\), the probability that the particle hits the state \(E\) before it hits \(B\) is \fbox{\(\frac{1}{4}\)}.
    \end{solution}
  \end{questions}
\end{document}
