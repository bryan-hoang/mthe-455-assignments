% Provides macros manipulating strings of tokens.
\RequirePackage{xstring}

% Store the jobname as a string with category 11 characters.
\edef\normaljobname{\expandafter\scantokens\expandafter{\jobname\noexpand}}
\StrBetween{\normaljobname}{asgmt-}{-q}[\assignmentnumber]
\StrBehind{\normaljobname}{-q-}[\questionnumber]

\documentclass[
  coursecode={MTHE 455},
  assignmentname={Assignment \assignmentnumber},
  studentnumber=20053722,
  name={Bryan Hoang},
  % final,
]{
  ltxanswer%
}

\usepackage{bch-style}

\begin{document}
  \begin{questions}
    \setcounter{question}{\questionnumber}
    \addtocounter{question}{-1}
    \question[5]\
    \begin{parts}
      \part{}
      \begin{solution}
        \begin{proof}
          \(\{X_{n}, n \ge 0\}\) has a one-step transition probability matrix
          \begin{equation*}
            P = \begin{bmatrix}
              p & 1-p & 0 & 0   \\
              0 & 0   & p & 1-p \\
              p & 1-p & 0 & 0   \\
              0 & 0   & p & 1-p
            \end{bmatrix}
          \end{equation*}
          Therefore, \(\{X_{n}, n \ge 0\}\) is a Markov chain.
        \end{proof}
      \end{solution}

      \part{}
      \begin{solution}
        Let \(\pi=(\pi_{0},\pi_{1},\pi_{2},\pi_{3})\). We have
        \begin{gather*}
          \pi = \pi P \\
          \Rightarrow \begin{cases}
            \pi_{0} = p\pi_{0} + p\pi_{2}         \\
            \pi_{1} = (1-p)\pi_{0} + (1-p)\pi_{2} \\
            \pi_{2} = p\pi_{1} + p\pi_{3}         \\
            \pi_{3} = (1-p)\pi_{1} + (1-p)\pi_{3} \\
          \end{cases}
        \end{gather*}
        and the additional constraint that
        \begin{equation*}
          \pi_{0} + \pi_{1} + \pi_{2} + \pi_{3} = 1
        \end{equation*}
        Solving the above system of equations gives
        \begin{equation*}
          \boxed{\pi = (p^{2}, p-p^{2}, p-p^{2}, p^{2}-2p+1)}
        \end{equation*}
      \end{solution}
    \end{parts}
  \end{questions}
\end{document}
