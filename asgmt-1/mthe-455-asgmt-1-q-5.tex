% Provides macros manipulating strings of tokens.
\RequirePackage{xstring}

% Store the jobname as a string with category 11 characters.
\edef\normaljobname{\expandafter\scantokens\expandafter{\jobname\noexpand}}
\StrBetween{\normaljobname}{asgmt-}{-q}[\assignmentnumber]
\StrBehind{\normaljobname}{-q-}[\questionnumber]

\documentclass[
  coursecode={MTHE 455},
  assignmentname={Assignment \assignmentnumber},
  studentnumber=20053722,
  name={Bryan Hoang}
]{
  ltxanswer%
}

\usepackage{bch-style}

\begin{document}
  \begin{questions}
    \setcounter{question}{\questionnumber}
    \addtocounter{question}{-1}
    \question{}
    \begin{solution}
      \begin{proof}
        For \(n \in \integers_{\ge 1}\), let \(A_{n} = \{X < s_{n}\}\). Since \(\{s_{n}:n\in\integers_{\ge 1}\}\) is an increasing sequence, then it follows that \(\{A_{n}:n \in \integers_{\ge 1}\}\) is an increasing sequence of events.
        \begin{claim}
          \(\cup_{n=1}^{\infty}A_n = \{X < s\}\)
        \end{claim}
        \begin{proof}
          For some \(x \in \cup_{n=1}^{\infty}A_n\), then \(\exists n \in \integers_{\ge 1}\) such that \(x < s_{n}\). But we also have \(s_{n} < s\), which implies that \(x < s\), and so \(x \in \{X < s\}\)

          For some \(x \in \{X < s\}\), then \(x < s \Rightarrow s-x>0\). Let \(\varepsilon = s-x\). Since \(s = \lim_{n \to \infty}s_{n}\), then \(\exists N \in \integers_{\ge 1}\) such that \(|s-s_{N}|<\varepsilon\). Then \(x < s_{N}\), so \(x \in \cup_{n=1}^{\infty}A_n\)
        \end{proof}
        It follows that
        \begin{align*}
          \lim_{n \to \infty} F(s_{n}) &= \lim_{n \to \infty} P(X \le s_{n})                                            \\
                                       &= \lim_{n \to \infty} P(A_{n})                                                  \\
                                       &= P(\lim_{n \to \infty} A_{n})       & &\text{by the continuity of probability} \\
                                       &= P(X < s)                           & &\text{by the previous claim}
        \end{align*}
      \end{proof}
    \end{solution}
  \end{questions}
\end{document}
