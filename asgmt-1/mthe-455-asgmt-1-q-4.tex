% Provides macros manipulating strings of tokens.
\RequirePackage{xstring}

% Store the jobname as a string with category 11 characters.
\edef\normaljobname{\expandafter\scantokens\expandafter{\jobname\noexpand}}
\StrBetween{\normaljobname}{asgmt-}{-q}[\assignmentnumber]
\StrBehind{\normaljobname}{-q-}[\questionnumber]

\documentclass[
  coursecode={MTHE 455},
  assignmentname={Assignment \assignmentnumber},
  studentnumber=20053722,
  name={Bryan Hoang}
]{
  ltxanswer%
}

\usepackage{bch-style}

\begin{document}
  \begin{questions}
    \setcounter{question}{\questionnumber}
    \addtocounter{question}{-1}
    \question{}
    \begin{solution}
      Let \(P(n,m)\) be the probability of interest. It has the following immediate properties based on the experiment:
      \begin{itemize}
        \item \(\forall (n,m) \in \integers_{\ge0} \times \integers_{\ge0}: n < m, P(n,m)=0\)
        \item \(\forall n \in \integers_{\ge 1}, P(n,0)=1\)
      \end{itemize}
      Then by using conditional probability,
      \begin{align*}
        P(n,m)             &= \frac{n}{n+m} P(n-1,m) + \frac{m}{n+m} P(n,m-1)\numberthis\label{eq:recurrence} \\
        \Rightarrow P(1,1) &= \frac{1}{2}                                                                     \\
        \Rightarrow P(1,1) &= \frac{1}{3}\left(2 \cdot \frac{1}{2} + 1\right) = \frac{2}{3}                   \\
        \Rightarrow P(2,2) &= \frac{1}{2} \cdot \frac{2}{3} = \frac{1}{3}
      \end{align*}

      \begin{table}
        \caption{The distribution of \(P(\cdot,\cdot)\)}\label{tab:distribution}
        \begin{tblr}{
            rows = {ht=32pt, rowsep = 2pt},
            columns = {wd=32pt, colsep = 2pt},
            cells = {m,c},
            hlines,
            vlines,
          }
          \diagbox[width=36pt,height=36pt]{\(n\)}{\(m\)} & 0       & 1               & 2               & \(\cdots\) \\
          0                                              & N/A     & 0               & 0               & 0\ldots    \\
          1                                              & 1       & \(\frac{1}{2}\) & 0               & 0\ldots    \\
          2                                              & 1       & \(\frac{2}{3}\) & \(\frac{1}{3}\) & \(\cdots\) \\
          \vdots                                         & 1\ldots & \vdots          & \vdots          & \(\ddots\)
        \end{tblr}
      \end{table}

      \newpage

      \begin{claim}
        \(P(n,m) = \frac{n-m+1}{n+1}\)
      \end{claim}
      \begin{proof}
        It is sufficient to show that the proposed probability satisfies~\eqref{eq:recurrence}.
        \begin{align*}
          P(n,m)       &= \frac{n}{n+m} P(n-1,m) + \frac{m}{n+m} P(n,m-1)                                             \\
                       &= \frac{n}{n+m} \cdot \frac{(n-1)-m+1}{(n-1)+1} + \frac{m}{n+m} \cdot \frac{n-(m-1)+1}{n+1}   \\
                       &= \frac{\cancel{n}}{n+m} \cdot \frac{n-m}{\cancel{n}} + \frac{m}{n+m} \cdot \frac{n-m+2}{n+1} \\
                       &= \frac{(n-m)(n+1) + m(n-m+2)}{(n+m)(n+1)}                                                    \\
                       &= \frac{n^{2} + n - \cancel{nm} - \cancel{m} + \cancel{nm} - m^{2} + \cancel{2}m}{(n+m)(n+1)} \\
                       &= \frac{(n+m)(n-m+1)}{(n+m)(n+1)}                                                             \\
          \alignedbox{ &= \frac{n-m+1}{n+1}}
        \end{align*}
      \end{proof}
    \end{solution}
  \end{questions}
\end{document}
