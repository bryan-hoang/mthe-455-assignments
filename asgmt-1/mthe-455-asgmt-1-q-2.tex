% Provides macros manipulating strings of tokens.
\RequirePackage{xstring}

% Store the jobname as a string with category 11 characters.
\edef\normaljobname{\expandafter\scantokens\expandafter{\jobname\noexpand}}
\StrBetween{\normaljobname}{asgmt-}{-q}[\assignmentnumber]
\StrBehind{\normaljobname}{-q-}[\questionnumber]

\documentclass[
  coursecode={MTHE 455},
  assignmentname={Assignment \assignmentnumber},
  studentnumber=20053722,
  name={Bryan Hoang}
]{
  ltxanswer%
}

\usepackage{bch-style}

\begin{document}
  \begin{questions}
    \setcounter{question}{\questionnumber}
    \addtocounter{question}{-1}
    \question\
    \begin{parts}
      \part{}
      \begin{solution}
        Let \(T_{i}\) be the time to collect the \(i\)-th coupon after \(i-1\) coupons
        have been collected. Then \(T = \sum_{i=1}^{n} T_{i}\). Observe that the
        probability of collecting a new coupon is \(p_{i} = \frac{n-(i-1)}{n} =
        \frac{n-i+1}{n}\). Therefore, \(T_{i}\) has a geometric distribution with
        expectation \(\frac{1}{p_{i}} = \frac{n}{n-i+1}\). By the linearity of
        expectations, we have:
        \begin{align*}
          \expect{T}   &= \expect{\sum_{i=1}^{n} T_{i}}  \\
                       &= \sum_{i=1}^{n} \expect{T_{i}}  \\
                       &= \sum_{i=1}^{n} \frac{n}{n-i+1} \\
                       &= n \sum_{i=1}^{n} \frac{1}{n}   \\
          \alignedbox{ &= n \cdot H_{n}}
        \end{align*}
        where \(H_{n}\) is the \(n\)-th harmonic number.
      \end{solution}

      \part{}
      \begin{solution}
        Let X be the numbers of times the coupon \(c_{1}\) appears in time
        \(T\). Then \(X\sim\mathrm{Binomial}(\frac{1}{n},T)\). Thus the
        probability we're interested in is
        \begin{equation*}
          \boxed{P(X=1) = \binom{T}{1} \frac{1}{n} \left(1-\frac{1}{n}\right)^{T-1}}
        \end{equation*}
      \end{solution}
    \end{parts}
  \end{questions}
\end{document}
