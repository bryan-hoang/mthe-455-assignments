% Provides macros manipulating strings of tokens.
\RequirePackage{xstring}

% Store the jobname as a string with category 11 characters.
\edef\normaljobname{\expandafter\scantokens\expandafter{\jobname\noexpand}}
\StrBetween{\normaljobname}{asgmt-}{-q}[\assignmentnumber]
\StrBehind{\normaljobname}{-q-}[\questionnumber]

\documentclass[
  coursecode={MTHE 455},
  assignmentname={Assignment \assignmentnumber},
  studentnumber=20053722,
  name={Bryan Hoang}
]{
  ltxanswer%
}

\usepackage{bch-style}

\begin{document}
  \begin{questions}
    \setcounter{question}{\questionnumber}
    \addtocounter{question}{-1}
    \question\
    \begin{parts}
      \part{}
      \begin{solution}
        Given the definition of T and the constraints of the experiment, the pmf of
        \(T\) can be derived as
        \begin{equation*}
          p_{T}(t) = \begin{cases}
            (p + q)^{t-1}r & \text{if } t \geq 1, \\
            0              & \text{otherwise},
          \end{cases}
        \end{equation*}
        Then the expected value of \(T\) is
        \begin{align*}
          \expect{T} &= \sum_{t=0}^{\infty} tp_{T}(t)       \\
                     &= \sum_{t=0}^{\infty} t(p + q)^{t-1}r
        \end{align*}
        Let \(x = p+q\). Then we have
        \begin{equation}\label{eq:integral}
          \int \expect{T} \dd{x} = \int \sum_{t=0}^{\infty} tx^{t-1}r \dd{x}
        \end{equation}
        \begin{claim}
          The series \(\sum_{t=1}^{\infty} t(x)^{t-1}r\) converges uniformly.
        \end{claim}
        \begin{proof}
          \begin{align*}
            L &= \lim_{t \to \infty} \frac{(t+1)(x)^{t}r}{t(x)^{t-1}r}                                   \\
              &= \lim_{t \to \infty} \frac{t + 1}{t}x                                                    \\
              &= x \lim_{t \to \infty} \frac{t + 1}{t}                                                   \\
              &= x                                                                                       \\
              &< 1                                                     & &\text{by the definition of } x
          \end{align*}
          Then by the ratio test, \(L < 1\) implies that the series converges uniformly.
        \end{proof}

        \newpage
        Then~\eqref{eq:integral} becomes

        \begin{align*}
          \int \expect{T} \dd{x}                    &= \sum_{t=0}^{\infty} \int tx^{t-1}r \dd{x} & &\because \text{the series converges uniformly} \\
                                                    &= r\sum_{t=0}^{\infty} x^{t}                                                                  \\
                                                    &= r \cdot \frac{1}{1-x}                     & &\because x < 1                                 \\
          \Rightarrow \dv{x} \int \expect{T} \dd{x} &= r \cdot \dv{x} \frac{1}{1-x}                                                                \\
          \expect{T}                                &= r \cdot \frac{1}{(1-x)^{2}}                                                                 \\
                                                    &= \frac{r}{r^2}                             & &\text{by the definition of } x                 \\
          \alignedbox{                              &= \frac{1}{r}}
        \end{align*}
      \end{solution}

      \part{}
      \begin{solution}
        Let \(X\) denote the number of \(\square{}\)'s that appear during time \(T\).
        Then the conditional probability of \(X\) given \(T\) is
        \begin{equation*}
          P(X=0|T=t) = p^{t-1}r
        \end{equation*}
        Then the probability that we want is
        \begin{align*}
          P(X=0)       &= \sum_{t=1}^{\infty} P(X=0|T=t)                   \\
                       &= \sum_{t=1}^{\infty} p^{t-1}r                     \\
                       &= r\sum_{t=0}^{\infty} p^{t-1}                     \\
          \alignedbox{ &= \frac{r}{1-p}}                 & &\because p < 1
        \end{align*}
      \end{solution}
    \end{parts}
  \end{questions}
\end{document}
