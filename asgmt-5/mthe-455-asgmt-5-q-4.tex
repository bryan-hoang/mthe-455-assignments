% region Filename parsing.
% Provides macros manipulating strings of tokens.
\RequirePackage{xstring}

% Store the jobname as a string with category 11 characters.
\edef\normaljobname{\expandafter\scantokens\expandafter{\jobname\noexpand}}
\StrBetween{\normaljobname}{asgmt-}{-q}[\assignmentnumber]
\StrBehind{\normaljobname}{-q-}[\questionnumber]
% endregion

\documentclass[
  coursecode={MTHE 455},
  assignmentname={Assignment \assignmentnumber},
  studentnumber=20053722,
  name={Bryan Hoang},
  draft,
  % final,
]{
  ltxanswer%
}

\usepackage{bch-style}

\begin{document}
  \begin{questions}
    \setcounter{question}{\questionnumber}
    \addtocounter{question}{-1}
    \question[6]\
    \begin{parts}
      \part{}
      \begin{solution}
        The continuous-time Markov chain \(\{X_{t} : t \ge 0\}\) has the state space \(I = \Z_{\ge 0} = \{0, 1, 2, \ldots\}\) with an associated \(Q\)-matrix defined by
        \begin{equation*}
          Q = \begin{cases}
            q_{0,1} = \lambda,\ q_{0,0} = -\lambda,                                                          \\
            q_{i,i-1} = i\mu,\ q_{i,i+1} = \lambda,\ q_{i,i} = -(i\mu + \lambda), &\forall i \in \Z_{\ge 1}, \\
            0,                                                                    &\text{otherwise.}
          \end{cases}
        \end{equation*}
      \end{solution}

      \part{}
      \begin{solution}
        To find the invariant distribution, \(\pi\), let's solve the detailed balance equation:
        \begin{align*}
          \pi_{i}q_{i,j}     &= \pi_{j}q_{j,i}                                                   & &\forall i,j \in I                                               \\
          \pi_{i-1}q_{i-1,i} &= \pi_{i}q_{i,i-1}                                                 & &\forall i \in \Z_{\ge 1}                                        \\
          \pi_{i-1}\lambda   &= \pi_{i}i\mu                                                      & &\forall i \in \Z_{\ge 1}                                        \\
          \pi_{i}            &= \frac{\lambda}{\mu} \cdot \frac{1}{i} \pi_{i-1}                  & &\forall i \in \Z_{\ge 1}                                        \\
          \pi_{i}            &= \biggl(\frac{\lambda}{\mu}\biggr)^{2} \frac{1}{i(i-1)} \pi_{i-2} & &\forall i \in \Z_{\ge 1}                                        \\
                             &\vdots                                                                                                                                \\
          \pi_{i}            &= \biggl(\frac{\lambda}{\mu}\biggr)^{i} \frac{1}{i!} \pi_{0}       & &\forall i \in \Z_{\ge 1}.\numberthis\label{eq:detailed-balance}
        \end{align*}
        To solve for \(\pi_{0}\), we know that \(\sum_{i=0}^{\infty} \pi_{i}=1\). Then summing~\eqref{eq:detailed-balance} over all \(i \in \Z_{\ge0}\) yields
        \begin{align*}
          \sum_{i=0}^{\infty} \pi_{i}                                                    &= \sum_{i=0}^{\infty} \biggl(\frac{\lambda}{\mu}\biggr)^{i} \frac{1}{i!} \pi_{0} \\
          \pi_{0} \sum_{i=0}^{\infty} \biggl(\frac{\lambda}{\mu}\biggr)^{i} \frac{1}{i!} &= 1                                                                              \\
          \pi_{0}                                                                        &= \frac{1}{\sum_{i=0}^{\infty} \bigl(\frac{\lambda}{\mu}\bigr)^{i} \frac{1}{i!}} \\
          \pi_{0}                                                                        &= e^{-\frac{\lambda}{\mu}}.
        \end{align*}
        Thus, \(\forall i \in \Z_{\ge0}\),
        \begin{equation*}
          \boxed{\pi_{i} = \frac{\bigl(\frac{\lambda}{\mu}\bigr)^{i}}{i!}e^{-\frac{\lambda}{\mu}}}.
        \end{equation*}
      \end{solution}
    \end{parts}
  \end{questions}
\end{document}
