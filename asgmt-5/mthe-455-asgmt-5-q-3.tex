% region Filename parsing.
% Provides macros manipulating strings of tokens.
\RequirePackage{xstring}

% Store the jobname as a string with category 11 characters.
\edef\normaljobname{\expandafter\scantokens\expandafter{\jobname\noexpand}}
\StrBetween{\normaljobname}{asgmt-}{-q}[\assignmentnumber]
\StrBehind{\normaljobname}{-q-}[\questionnumber]
% endregion

\documentclass[
  coursecode={MTHE 455},
  assignmentname={Assignment \assignmentnumber},
  studentnumber=20053722,
  name={Bryan Hoang},
  draft,
  % final,
]{
  ltxanswer%
}

\usepackage{bch-style}

\begin{document}
  \begin{questions}
    \setcounter{question}{\questionnumber}
    \addtocounter{question}{-1}
    \question[6]{}
    \begin{solution}
      To find the semigroup of \(Q\) (\(P(t)\)), first note that because of symmetry, let
      \begin{equation*}
        f(t) = \P_{i}{X_{t} = i}, \ \forall i \in I.
      \end{equation*}
      Then for \(i \ne j\) and since \(\forall i \in I, t \in \R_{>0}, \sum_{j \in I} P_{i,j}(t) = 1\),
      \begin{equation*}
        \P_{i}{X_{t} = j} = \frac{1 - f(t)}{2}.
      \end{equation*}
      Thus, the semigroup of \(Q\) is
      \begin{equation*}
        P(t) = \begin{bmatrix}
          f(t)             & \frac{1-f(t)}{2} & \frac{1-f(t)}{2} \\
          \frac{1-f(t)}{2} & f(t)             & \frac{1-f(t)}{2} \\
          \frac{1-f(t)}{2} & \frac{1-f(t)}{2} & f(t)
        \end{bmatrix}.
      \end{equation*}
      Since \(\forall i \ne j,\ Q_{i,j} = 1\) and \(\forall i \in I,\ \sum_{j \in I} Q_{i,j} = 0\), the \(Q\)-matrix is
      \begin{equation*}
        Q = \begin{bmatrix}
          -2 & 1  & 1  \\
          1  & -2 & 1  \\
          1  & 1  & -2
        \end{bmatrix}.
      \end{equation*}
      To solve for \(f(t)\), we can use the backward equation, which states that
      \begin{equation}\label{eq:backward}
        P'(t) = QP(t);\quad P(0) = I_{3},
      \end{equation}
      where \(I_{3}\) is the identity matrix of size \(3\). Then~\eqref{eq:backward} gives
      \begin{equation}\label{eq:backward-implication}
        f'(t) = -2f(t) + \frac{1 - f(t)}{2} + \frac{1 - f(t)}{2};\quad f(0) = 1.
      \end{equation}
      Then
      \begin{align*}
        f'(t)                                         &= -3f(t) + 1                                                      \\
        \Rightarrow \biggl(f(t) - \frac{1}{3}\biggr)' &= -3\biggl(f(t) - \frac{1}{3}\biggr).\numberthis\label{eq:diffeq}
      \end{align*}
      Solving~\eqref{eq:diffeq} for \(f(t)\) gives
      \begin{equation*}
        f(t) = \frac{1}{3} + Ce^{-3t}
      \end{equation*}
      where \(C \in \R\) is an unknown constant. Since by~\eqref{eq:backward-implication}, \(f(0) = 1\), then \(C = \frac{2}{3}\). Therefore,
      \begin{equation*}
        \renewcommand{\arraystretch}{1.5}
        \boxed{
          P(t) = \begin{bmatrix*}[l]
            \frac{1}{3} (1 + 2e^{-3t}) & \frac{1}{3} (1 - e^{-3t})  & \frac{1}{3} (1 - e^{-3t})  \\
            \frac{1}{3} (1 - e^{-3t})  & \frac{1}{3} (1 + 2e^{-3t}) & \frac{1}{3} (1 - e^{-3t})  \\
            \frac{1}{3} (1 - e^{-3t})  & \frac{1}{3} (1 - e^{-3t})  & \frac{1}{3} (1 + 2e^{-3t})
          \end{bmatrix*}
        }.
      \end{equation*}
    \end{solution}
  \end{questions}
\end{document}
