% Provides macros manipulating strings of tokens.
\RequirePackage{xstring}

% Store the jobname as a string with category 11 characters.
\edef\normaljobname{\expandafter\scantokens\expandafter{\jobname\noexpand}}
\StrBetween{\normaljobname}{asgmt-}{-q}[\assignmentnumber]
\StrBehind{\normaljobname}{-q-}[\questionnumber]

\documentclass[
  coursecode={MTHE 455},
  assignmentname={Assignment \assignmentnumber},
  studentnumber=20053722,
  name={Bryan Hoang},
  draft,
  % final,
]{
  ltxanswer%
}

\usepackage{bch-style}

\begin{document}
  \begin{questions}
    \setcounter{question}{\questionnumber}
    \addtocounter{question}{-1}
    \question[5]{}
    \begin{solution}
      \begin{claim}
        \begin{equation*}
          \lim_{n \to \infty} p_{n} = \frac{\alpha_{1}\alpha_{2}}{\alpha_{2}+\alpha_{3}}
        \end{equation*}
      \end{claim}
      \begin{proof}
        Let \(X_{n}\) denote the ordering of the books on day \(n\) with state space
        \begin{equation*}
          S=\{123, 132, 213, 231, 312, 321\}.
        \end{equation*}
        Then \((X_{n})_{\Z_{\ge 0}}\) forms a Markov chain. Then the transition matrix is
        \begin{equation*}
          P = \begin{bmatrix}
            \alpha_{1} & 0          & \alpha_{2} & 0          & \alpha_{3} & 0          \\
            0          & \alpha_{1} & \alpha_{2} & 0          & \alpha_{3} & 0          \\
            \alpha_{1} & 0          & \alpha_{2} & 0          & 0          & \alpha_{3} \\
            \alpha_{1} & 0          & 0          & \alpha_{2} & 0          & \alpha_{3} \\
            0          & \alpha_{1} & 0          & \alpha_{2} & \alpha_{3} & 0          \\
            0          & \alpha_{1} & 0          & \alpha_{2} & 0          & \alpha_{3} \\
          \end{bmatrix}
        \end{equation*}
        It is clear that \(P\) is irreducible.

        The state \(123 \in S\) is aperiodic since \(P_{123,123}^{n} = \alpha_{1}^{n} > 0, \forall n \in \Z_{\ge 1}\). Given that and the fact that \(P\) is irreducible, we can see that \(P\) is also aperiodic.

        Since \(P\) is irreducible and aperiodic, then the theorem on convergence to equilibrium tells us that
        \begin{equation*}
          \lim_{n \to \infty} p_{n} = \pi_{123}
        \end{equation*}
        where \(\pi_{123}\) is a component of the unique invariant distribution of \(P\).

        The unique invariant distribution of \(P\) satisfies
        \begin{equation*}
          \pi = \pi P
        \end{equation*}
        which solving gives
        \begin{equation*}
          \pi = \biggl(\frac{\alpha_{1}\alpha_{2}}{\alpha_{2}+\alpha_{3}},\frac{\alpha_{1}\alpha_{3}}{\alpha_{2}+\alpha_{3}},\frac{\alpha_{1}\alpha_{2}}{\alpha_{1}+\alpha_{3}},\frac{\alpha_{2}\alpha_{3}}{\alpha_{1}+\alpha_{3}},\frac{\alpha_{1}\alpha_{2}}{\alpha_{1}+\alpha_{2}},\frac{\alpha_{2}\alpha_{3}}{\alpha_{1}+\alpha_{2}}\biggr)
        \end{equation*}
        Therefore,
        \begin{align*}
          \lim_{n \to \infty} p_{n} &= \pi_{123}                                           \\
          \alignedbox{              &= \frac{\alpha_{1}\alpha_{2}}{\alpha_{2}+\alpha_{3}}}
        \end{align*}
      \end{proof}
    \end{solution}
  \end{questions}
\end{document}
