% Provides macros manipulating strings of tokens.
\RequirePackage{xstring}

% Store the jobname as a string with category 11 characters.
\edef\normaljobname{\expandafter\scantokens\expandafter{\jobname\noexpand}}
\StrBetween{\normaljobname}{asgmt-}{-q}[\assignmentnumber]
\StrBehind{\normaljobname}{-q-}[\questionnumber]

\documentclass[
  coursecode={MTHE 455},
  assignmentname={Assignment \assignmentnumber},
  studentnumber=20053722,
  name={Bryan Hoang},
  draft,
  % final,
]{
  ltxanswer%
}

\usepackage{bch-style}

\begin{document}
  \begin{questions}
    \setcounter{question}{\questionnumber}
    \addtocounter{question}{-1}
    \question[6]\
    \begin{parts}
      \part{}
      \begin{solution}
        \begin{claim}
          The invariant distribution of \(P\) is
          \begin{equation*}
            \pi = \Bigl(\underbrace{\frac{1}{N+1},\dotsc,\frac{1}{N+1}}_{N+1\ \text{components}}\Bigr).
          \end{equation*}
        \end{claim}
        \begin{proof}
          The invariant distribution should satisfy
          \begin{align*}
            \pi                    &= \pi P                                                                                          \\
            \pi_{j}                &= \sum_{i \in S} \pi_{i} P_{ij}, \quad \forall j \in S\numberthis\label{eq:invariant-transition} \\
            \intertext{and}
            \sum_{i \in S} \pi_{i} &= 1.\numberthis\label{eq:invariant-sum}
          \end{align*}
          Then starting with the RHS of~\eqref{eq:invariant-transition}, we have that \(\forall j \in S\),
          \begin{align*}
            \sum_{i \in S} \pi_{i} P_{ij} &= \frac{1}{N+1} \sum_{i \in S} P_{ij}                                                          \\
                                          &= \frac{1}{N+1} \sum_{i \in S} P_{ji} & &\text{by the definition of \(P\)}                     \\
                                          &= \frac{1}{N+1}                       & &\because P\ \text{is always a left stochastic matrix} \\
                                          &= \pi_{j}.
          \end{align*}
          Starting with the LHS of~\eqref{eq:invariant-sum} gives
          \begin{align*}
            \sum_{i \in S} \pi_{i} &= \sum_{i \in S} \frac{1}{N+1} \\
                                   &= \frac{1}{N+1} (N+1)          \\
                                   &= 1.
          \end{align*}
          Since \(P\) is also irreducible, we can conclude that \(\pi\) is the unique invariant distribution of \(P\).
        \end{proof}
      \end{solution}

      \part{}
      \begin{solution}
        \begin{claim}
          The invariant distribution of \(P\) is defined by the components
          \begin{equation*}
            \pi_{i} = \frac{\sum_{k \in S} w_{ik}}{\sum_{l \in S} \sum_{m \in S} w_{lm}}.
          \end{equation*}
        \end{claim}
        \begin{proof}
          The invariant distribution should satisfy
          \begin{align*}
            \pi                    &= \pi P                                                                                          \\
            \pi_{j}                &= \sum_{i \in S} \pi_{i} P_{ij}, \quad \forall j \in S\numberthis\label{eq:invariant-transition} \\
            \intertext{and}
            \sum_{i \in S} \pi_{i} &= 1.\numberthis\label{eq:invariant-sum}
          \end{align*}
          Then starting with the RHS of~\eqref{eq:invariant-transition}, we have that \(\forall j \in S\),
          \begin{align*}
            \sum_{i \in S} \pi_{i} P_{ij} &= \sum_{i \in S} \frac{\sum_{k \in S} w_{ik}}{\sum_{l \in S} \sum_{m \in S} w_{lm}} \cdot \frac{w_{ij}}{\sum_{k \in S} w_{ik}} \\
                                          &=  \frac{\sum_{i \in S} w_{ij}}{\sum_{l \in S} \sum_{m \in S} w_{lm}}                                                          \\
                                          &=  \frac{\sum_{k \in S} w_{jk}}{\sum_{l \in S} \sum_{m \in S} w_{lm}}                                                          \\
                                          &= \pi_{j}.
          \end{align*}
          Starting with the LHS of~\eqref{eq:invariant-sum} gives
          \begin{align*}
            \sum_{i \in S} \pi_{i} &= \sum_{i \in S} \frac{\sum_{k \in S} w_{ik}}{\sum_{l \in S} \sum_{m \in S} w_{lm}} \\
                                   &= \frac{\sum_{i \in S} \sum_{k \in S} w_{ik}}{\sum_{l \in S} \sum_{m \in S} w_{lm}} \\
                                   &= 1.
          \end{align*}
          Since \(P\) is also irreducible, we can conclude that \(\pi\) is the unique invariant distribution of \(P\).
        \end{proof}
      \end{solution}

      \part{}
      \begin{solution}
        \begin{proof}
          42.
        \end{proof}
      \end{solution}
    \end{parts}
  \end{questions}
\end{document}
