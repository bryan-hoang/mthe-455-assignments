% Provides macros manipulating strings of tokens.
\RequirePackage{xstring}

% Store the jobname as a string with category 11 characters.
\edef\normaljobname{\expandafter\scantokens\expandafter{\jobname\noexpand}}
\StrBetween{\normaljobname}{asgmt-}{-q}[\assignmentnumber]
\StrBehind{\normaljobname}{-q-}[\questionnumber]

\documentclass[
  coursecode={MTHE 455},
  assignmentname={Assignment \assignmentnumber},
  studentnumber=20053722,
  name={Bryan Hoang},
  draft,
  % final,
]{
  ltxanswer%
}

\usepackage{bch-style}

\begin{document}
  \begin{questions}
    \setcounter{question}{\questionnumber}
    \addtocounter{question}{-1}
    \question[5]\
    \begin{parts}
      \part{}
      \begin{solution}
        \begin{claim}
          The invariant distribution of \(P\) is
          \begin{equation*}
            \pi = \Bigl(\underbrace{\frac{1}{M+1},\dotsc,\frac{1}{M+1}}_{M+1\ \text{components}}\Bigr).
          \end{equation*}
        \end{claim}
        \begin{proof}
          The invariant distribution should satisfy
          \begin{align*}
            \pi                    &= \pi P                                                                                          \\
            \pi_{j}                &= \sum_{i \in S} \pi_{i} P_{ij}, \quad \forall j \in S\numberthis\label{eq:invariant-transition} \\
            \intertext{and}
            \sum_{i \in S} \pi_{i} &= 1.\numberthis\label{eq:invariant-sum}
          \end{align*}
          Then starting with the RHS of~\eqref{eq:invariant-transition}, we have that \(\forall j \in S\),
          \begin{align*}
            \sum_{i \in S} \pi_{i} P_{ij} &= \frac{1}{M+1} \sum_{i \in S} P_{ij}                                                   \\
                                          &= \frac{1}{M+1}                       & &\because P\ \text{is  right stochastic matrix} \\
                                          &= \pi_{j}.
          \end{align*}
          Starting with the LHS of~\eqref{eq:invariant-sum} gives
          \begin{align*}
            \sum_{i \in S} \pi_{i} &= \sum_{i \in S} \frac{1}{M+1} \\
                                   &= \frac{1}{M+1} (M+1)          \\
                                   &= 1.
          \end{align*}
          Since \(P\) is also irreducible, we can conclude that \(\pi\) is the unique invariant distribution of \(P\).
        \end{proof}
      \end{solution}

      \part{}
      \begin{solution}
        Let \(Y_{n}= X_{n} \bmod 11\). Then \((Y_{n})_{\Z_{\ge 0}}\) is a Markov chain with the state space \(S=\{0,1,\dotsc,10\}\) and transition matrix
        \begin{equation*}
          P = \begin{bmatrix}
            0           & \frac{1}{6} & \frac{1}{6} & \frac{1}{6} & \frac{1}{6} & \frac{1}{6} & \frac{1}{6} & 0           & 0           & 0           & 0           \\[1ex]
            0           & 0           & \frac{1}{6} & \frac{1}{6} & \frac{1}{6} & \frac{1}{6} & \frac{1}{6} & \frac{1}{6} & 0           & 0           & 0           \\[1ex]
            0           & 0           & 0           & \frac{1}{6} & \frac{1}{6} & \frac{1}{6} & \frac{1}{6} & \frac{1}{6} & \frac{1}{6} & 0           & 0           \\[1ex]
            0           & 0           & 0           & 0           & \frac{1}{6} & \frac{1}{6} & \frac{1}{6} & \frac{1}{6} & \frac{1}{6} & \frac{1}{6} & 0           \\[1ex]
            0           & 0           & 0           & 0           & 0           & \frac{1}{6} & \frac{1}{6} & \frac{1}{6} & \frac{1}{6} & \frac{1}{6} & \frac{1}{6} \\[1ex]
            \frac{1}{6} & 0           & 0           & 0           & 0           & 0           & \frac{1}{6} & \frac{1}{6} & \frac{1}{6} & \frac{1}{6} & \frac{1}{6} \\[1ex]
            \frac{1}{6} & \frac{1}{6} & 0           & 0           & 0           & 0           & 0           & \frac{1}{6} & \frac{1}{6} & \frac{1}{6} & \frac{1}{6} \\[1ex]
            \frac{1}{6} & \frac{1}{6} & \frac{1}{6} & 0           & 0           & 0           & 0           & 0           & \frac{1}{6} & \frac{1}{6} & \frac{1}{6} \\[1ex]
            \frac{1}{6} & \frac{1}{6} & \frac{1}{6} & \frac{1}{6} & 0           & 0           & 0           & 0           & 0           & \frac{1}{6} & \frac{1}{6} \\[1ex]
            \frac{1}{6} & \frac{1}{6} & \frac{1}{6} & \frac{1}{6} & \frac{1}{6} & 0           & 0           & 0           & 0           & 0           & \frac{1}{6} \\[1ex]
            \frac{1}{6} & \frac{1}{6} & \frac{1}{6} & \frac{1}{6} & \frac{1}{6} & \frac{1}{6} & 0           & 0           & 0           & 0           & 0
          \end{bmatrix}
        \end{equation*}
        which is an irreducible, aperiodic, and right stochastic matrix. Then by part~(\ref{part@2@1}), with \(M=10\), the invariant distribution of \(P\) is
        \begin{equation*}
          \pi = \Bigl(\underbrace{\frac{1}{11},\dotsc,\frac{1}{11}}_{11\ \text{components}}\Bigr).
        \end{equation*}
        Since the matrix is irreducible and aperiodic, the convergence to equilibrium theorem says that
        \begin{align*}
          \lim_{n \to \infty} P(X_{n}\ \text{is a multiple of}\ 11) &= \lim_{n \to \infty} P(Y_{n}=0) \\
                                                                    &= \pi_{0}                        \\
          \alignedbox{                                              &= \frac{1}{11}}.
        \end{align*}
      \end{solution}
    \end{parts}
  \end{questions}
\end{document}
